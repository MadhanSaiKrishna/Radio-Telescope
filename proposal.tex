\documentclass[12pt]{report}
\usepackage[a4paper,margin=1in]{geometry}
\usepackage{lmodern}
\usepackage{setspace}
\usepackage{hyperref}
\usepackage{graphicx}
\usepackage{titlesec}
\usepackage{enumitem}
\usepackage{longtable}
\usepackage[table]{xcolor}
\usepackage{array}
\usepackage[absolute,overlay]{textpos}


\begin{document}

\begin{titlepage}
    \begin{flushleft}
    \begin{textblock*}{5cm}(1cm,1cm) % Position: 1cm from left and 1cm from top
        \includegraphics[width=3cm]{Astro_Club_logo.png} % Adjust width as needed
    \end{textblock*}
\end{flushleft}

\begin{flushright}
    \begin{textblock*}{5cm}(15cm,1cm) % Position: 1cm from left and 1cm from top
        \includegraphics[width=3cm]{Astro_Club_logo.png} % Adjust width as needed
    \end{textblock*}
\end{flushright}

\centering
\vspace{5em} % Adds vertical space above the heading (adjust as needed)
{\huge \textbf{Radio Telescope}}\\[1em]
{\normalsize \textbf{Astronautics Club - Project Proposal}}\\
{\normalsize International Institute of Information Technology, Hyderabad}\\[5em]

\begin{flushleft}
    \textit{Respected Sir/Ma’am,}

    \medskip

    The Astronautics Club plans to start a new project – building a functional Radio Telescope. This hands-on initiative will provide students with practical experience in radio astronomy, electronics, data analysis, and collaborative problem-solving.

    \medskip

    Our goal is to build a low-cost, small-scale radio telescope capable of detecting celestial radio emissions. Initially, the telescope will focus on strong radio sources like the Sun and progress to more advanced observations.

    \medskip

    We will use the \href{https://github.com/astrochart/CHART}{CHART (Completely Hackable Radio Telescope)} model with our own optimizations. The initial design will include:
    \begin{itemize}[leftmargin=1.5em]
        \item A parabolic dish antenna with a feedhorn system for focusing incoming radio waves
        \item A Low Noise Amplifier (LNA) and bandpass filter to enhance weak celestial signals
        \item A Software-Defined Radio (SDR) receiver for flexible signal processing
        \item Data acquisition and processing software using \href{https://github.com/gnuradio}{GNU Radio} and Python-based tools
    \end{itemize}

    In the following sections, we share our implementation timeline and potential future improvements. We welcome suggestions and collaborations to make this project a valuable learning experience for all involved.

    \medskip

    \textbf{Regards,} \\
    Madhan Sai Krishna \\
    Co-Coordinator, Astronautics Club
    \end{flushleft}
\end{titlepage}
    \vspace{2em}
    \begin{flushleft}
    {\Large \textbf{Implementation Timeline}}\\[1em]

    \textbf{Phase 1: Research \& Planning}
    \begin{itemize}[leftmargin=1.5em]
        \item Study radio telescope designs, focusing on HI-line detection and planetary emissions
        \item Develop a detailed blueprint for the antenna, receiver, and data pipeline
        \item Estimate budget and procure required components
    \end{itemize}

    \textbf{Phase 2: Construction \& Assembly}
    \begin{itemize}[leftmargin=1.5em]
        \item Build the parabolic dish and feedhorn system
        \item Assemble the LNA, filters, and SDR receiver setup
        \item Perform initial system integration and debugging
    \end{itemize}

    \textbf{Phase 3: Testing \& Calibration}
    \begin{itemize}[leftmargin=1.5em]
        \item Test the system using terrestrial radio sources
        \item Conduct celestial tests and compare signals with public datasets
        \item Calibrate the telescope for accurate measurements
        \item Optimize performance through iterative refinements
    \end{itemize}

    \textbf{Phase 4: Data Collection \& Analysis}
    \begin{itemize}[leftmargin=1.5em]
        \item Implement data processing techniques to extract meaningful information
        \item Document findings and prepare presentations for the club
        \item Schedule and conduct observation sessions, beginning with the Sun
    \end{itemize}

    \section*{To-do:}
    \begin{itemize}
        \item Add the budget and components list
        \item 
    \end{itemize}
    \end{flushleft}


\end{document}
