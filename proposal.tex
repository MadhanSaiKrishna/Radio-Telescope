\documentclass[12pt]{report}
\usepackage[a4paper,margin=1in]{geometry}
\usepackage{lmodern}
\usepackage{setspace}
\usepackage{hyperref}
\usepackage{graphicx}
\usepackage{titlesec}
\usepackage{enumitem}
\usepackage{longtable}
\usepackage[table]{xcolor}
\usepackage{array}
\usepackage[absolute,overlay]{textpos}
\usepackage{hyperref}
\usepackage{url}
\usepackage{makecell}

\sloppy
\def\UrlBreaks{\do\/\do-}



\begin{document}

\begin{titlepage}
    \begin{flushleft}
    \begin{textblock*}{5cm}(1.5cm,1cm) 
        \includegraphics[width=3.75cm]{Astro_Club_logo.png} 
    \end{textblock*}
\end{flushleft}

\begin{flushright}
    \begin{textblock*}{5cm}(15.5cm,1.5cm) 
        \includegraphics[width=5cm]{iiith_logo.png}
    \end{textblock*}
\end{flushright}

\centering

{\huge \textbf{Radio Telescope}}\\[1em]
{\normalsize \textbf{Astronautics Club - Project Proposal}}\\
{\normalsize International Institute of Information Technology, Hyderabad}\\[5em]

\begin{flushleft}
    \textit{Respected Sir/Ma’am,}

    \medskip

    We, the members of the Astronautics Club, are excited to propose a project to build a small-scale low-cost radio telescope. This project aims to provide hands-on experience in a range of inter-disciplinary domains, including electronics, computer science, signal proocessingn, and radio astronomy.  

    \medskip
    % \textbf{Project Overview:} \\
    \section*{Project Overview:}
    Through this project, we aim to build a small-scale radio telescope that can be assembled and opererated by students primarily for experimental and educational purposes. \\
     The initial objective is to detect strong and reliable sources such as the Sun, the artificial satellites, the Milky Way, etc. As the project progresses, we aim to improve the sensitivity and resolution of the telescope to detect weaker sources.\\

    \medskip

    Our design and methodology will be guided by well-documented open-source radio telescope projects, like - CHART \cite{ref1} and Project RT \cite{ref4}\\
    The telescope system will consist of the following modules:
    \begin{itemize}[leftmargin=1.5em]
        \item Antenna system: A parabolic dish antenna \& a custom feedhorn
        \item Signal conditioning chain: LNA \& bandpass filters
        \item A Software-Defined Radio (SDR) receiver: RTL-SDR with a dipole antenna
        \item Data acquisition and processing System :  \href{https://github.com/gnuradio}{GNU Radio} and Python-based tools
    \end{itemize}

    \medskip
    % \textbf{Project Impact:} \\
    \section*{Project Impact:}
    \begin{itemize}
        \item Foster interdisciplinary collaboration among students in electronics, computer science, and radio astronomy
        \item Develop hands-on experience in antenna theory and design
        \item Enable exploration of signal processing techniques using tools like GNU Radio and Python
        \item Enhance research and analytical skills through data collection, processing, etc. 
    \end{itemize}

    In the following sections, we share our implementation timeline and potential future improvements. We welcome suggestions and collaborations to make this project a valuable learning experience for all involved.

    \medskip
    \end{flushleft}
\end{titlepage}

\section*{Key Components: }
\begin{enumerate}
    \item \textbf{Antenna System:} 
    \begin{itemize}
        \item A parabolic dish antenna (e.g. a repurposed satellite dish) - primary collector of incoming radio waves
        \item A custom feedhorn - to direct signals to the receiver chain
    \end{itemize}
    \item \textbf{Signal Conditioning Chain:} 
    \begin{itemize}
        \item LNA - to amplify weak celestial signals and minimize other electronic noises
        \item Bandpass filter - To isolate our desired frequency band (1.42GHz - 21cm Hydrogen line)
    \end{itemize}
    \item \textbf{Software-Defined Radio (SDR) Receiver}
    \begin{itemize}
        \item RTL-SDR with custom antenna - to convert incoming analog signals into a format suitable fro proccessing
    \end{itemize}
    \item \textbf{Data Acquisition and Processing System:}
    \begin{itemize}
        \item GNU Radio, CubicSDR, etc. for real time signal processing
        \item Python based tools for computational Analysis
        \item MATLAB, SciPy, etc. for signal visualization
    \end{itemize}
\end{enumerate}
    % \vspace{2em}
    \begin{flushleft}

\section*{Project Timeline:}
    \textbf{Phase 0: Literature review and Component procurement}
    \begin{itemize}[leftmargin=1.5em]
        \item Literature review of existing amateur radio telescope builds 
        \item Finalize bill of materials and begin procurement of key components
        \item JDocument the specifications of the project
    \end{itemize}
    \textbf{Phase 1: Research \& Planning}
    \begin{itemize}[leftmargin=1.5em]
        \item Study radio telescope designs, focusing on HI-line detection and planetary emissions
        \item Develop a detailed blueprint for the antenna, receiver, and data pipeline
        \item Estimate budget and procure required components
    \end{itemize}

    \textbf{Phase 2: Construction \& Assembly}
    \begin{itemize}[leftmargin=1.5em]
        \item Build the parabolic dish and feedhorn system
        \item Assemble the LNA, filters, and SDR receiver setup
        \item Perform initial system integration and debugging
    \end{itemize}

    \textbf{Phase 3: Testing \& Calibration}
    \begin{itemize}[leftmargin=1.5em]
        \item Test the system using terrestrial radio sources
        \item Conduct celestial tests and compare signals with public datasets
        \item Calibrate the telescope for accurate measurements
        \item Optimize performance through iterative refinements
    \end{itemize}

    \textbf{Phase 4: Data Collection \& Analysis}
    \begin{itemize}[leftmargin=1.5em]
        \item Implement data processing techniques to extract meaningful information
        \item Document findings and prepare presentations for the club
        \item Schedule and conduct observation sessions, beginning with the Sun
    \end{itemize}
    \end{flushleft}
\section*{Bill of Materials: Radio Telescope Project}
The following table outlines the estimated components required to construct the small-scale radio telescope. 
\begin{table}[h]
    \centering
    \resizebox{\textwidth}{!}{% Resize table to fit page width
        \begin{tabular}{|c|c|c|c|}
            \hline
            \rowcolor{gray!20}
            \textbf{Component} & \textbf{Specification} & \textbf{Estimated Cost (INR)} & \textbf{Supplier Link} \\
            \hline
            Aluminium Foil adhesive tape & - & 200 & \href{https://amzn.in/d/7cCpcly
            }{Amazon} \\
            \hline
            Aluminium Foil & - & 400 & \href{https://amzn.in/d/i8KKOHn
            }{Amazon}\\
            \hline
            Cardboard Sheets & 4 36" x 48" & 500 &  \href{ https://amzn.in/d/8RtQvZT}{Amazon}\\
            \hline
            Coaxial Cable & - & - & \href{https://amzn.in/d/a2lSafl}{Amazon} / Local\\
            \hline
            SMA female connector & - & 790 & \href{https://amzn.in/d/hAuZsYu}{Amazon}\\
            \hline
            SMA male to male right angle connector & - & - & \href{https://amzn.in/d/cBtrowx}{Amazon}\\
            \hline
            male to male \& female to female connectors & - & - & \href{https://amzn.in/d/dfnezPs}{Amazon}\\
            \hline
            RTL-SDR with Dipole Antenna Kit & - & & \href{ https://amzn.in/d/bKsKqL1}{Amazon}\\
            \hline
            RTL Dongle & - & 8450 & \href{https://amzn.in/d/brW76fM}{Amazon}\\
            \hline
            \makecell{TATA Sky Set up (Reflector Parabolic Dish,\\ the mount attachment, LNB and set-top box)} & - & 5000 & Link\\
            \hline

        \end{tabular}%
    }
    \caption{Estimated costs}
    % \label{tab:components}
\end{table}
\section*{To-do:}
\begin{itemize}
    \item complete the bill of materials table with actual working and reliable links from which we can procure the components
    \item format the doc
    \item articulate the project timeline in a better way and include more details about the key cocmponents
    \item add few images of what can be expected from the project (use Project RT iamges as reference)
    \item highlight the research benefits 
    \item 
\end{itemize}

\begin{thebibliography}{9}
\bibitem{ref1}
CHART (Completely Hackable Radio Telescope) \\
\url{https://astrochart.github.io/telescope_design}
\bibitem{ref2}
NASA's Radio Jove Project \\
\url{https://radiojove.gsfc.nasa.gov/radio_telescope/building_testing.php}
\bibitem{ref3}
CBSS Nsukka \\
\url{https://astro4dev.org/nigerian-project-develops-diy-low-cost-radio-telescope-for-teaching/}
\bibitem{ref4}
Project Radio Telescope - BITS Goa \\
% \url{https://github.com/Project-RT-BPGC/CompSci}
\url{https://projectrtbits.wordpress.com/}
\bibitem{ref5}
MIT Haystack Observatory \\
\url{https://www.haystack.mit.edu/haystack-public-outreach/srt-the-small-radio-telescope-for-education/}
\bibitem{ref6}
IEEE Penn State Harisburg \\
\url{https://edu.ieee.org/us-psu/2024/01/20/radio-telescope-project/}
\end{thebibliography}
\begin{flushleft}
    Thank you for considering our proposal. We look forward to your support and are happy to provide any additional details or clafications as required. \\ \vspace{1em}

    Sincerely,\\ 
    Name \\
    On behalf of the Astronautics Club\\
    Department and Year \\
    Contact info
\end{flushleft}


\end{document}